%%=============================================================================
%% Inleiding
%%=============================================================================

\chapter{\IfLanguageName{dutch}{Inleiding}{Introduction}}%
\label{ch:inleiding}

Veel bedrijven hebben wat men noemt 'een vuilnistabel'. Dit is een tabel met ongestructureerde masterdata. Hierin staan allerlei gegevens door elkaar, waardoor het lastig wordt om iets te doen met deze data. In dit onderzoek zal er gekeken worden naar de verschillende mogelijkheden om data-extractie te verbeteren in zo'n tabellen. Data-extractie of gegevensextractie slaat op het ophalen van relevante gegevens uit een tabel of databank \autocite{Encyclo.nl}. Aan de hand van dit onderzoek kunnen bedrijven hun vuilnistabellen veel beter gebruiken, of zelfs sorteren, zonder alles handmatig uit te voeren.
\\\indent
In dit onderzoek wordt enkel masterdata onderzocht. Masterdata zijn de gegevens van een bedrijf die niet transactioneel zijn. Dit betekent dat de data niet te maken heeft met transacties zoals het plaatsen van orders bijvoorbeeld. Masterdata kan wel veranderen in de loop der tijd, maar dit gebeurt zeer langzaam. Eén van de voornaamste voorbeelden hiervan zijn productgegevens: de naam, afmetingen, prijs... Deze gegevens gaan enkel aangepast worden in uitzonderlijke gevallen \autocite{Yellowground}.
\\\indent
Er zijn twee mogelijkheden om deze data op te slaan: gestructureerd of ongestructureerd. Gestructureerde data is georganiseerd en gegroepeerd zodat alle verschillende gegevens in een aparte tabel zitten. Als dit niet het geval is en er bestaat slechts één tabel met daarin allerlei gegevens, is er sprake van ongestructureerde data. Een voorbeeld hiervan is een tabel met de naam, prijs, afmetingen, beschrijving van het product allemaal tezamen \autocite{Seagate}.
\\\indent
Het grote probleem met ongestructureerde masterdata is het feit dat het verbeteren van de datakwaliteit een stuk moeilijker wordt. Zoeken of sorteren in zo'n tabel is heel moeilijk, ook duplicaten onderscheiden is niet gemakkelijk. Daarnaast kan het soms zelfs lastig worden om de gewenste data op te halen uit zo'n tabel.
\\\indent
In dit onderzoek zullen er verschillende mogelijkheden vergeleken worden om gelijkaardige waarden uit een tabel te vinden, beter gekend als fuzzymatching \autocite{Silva2022}. Er bestaan al een heleboel technieken en metrieken hiervoor, maar wat als we de tabel eerst gaan onderverdelen in verschillende groepen (Clustering) waarbij het de bedoeling is dat elke groep iets gelijkaardigs heeft zodat er een label aan iedere groep kan gegeven worden (Tagging)? De methode clustering + tagging gaat onderzocht en vergeleken worden met bestaande algoritmes die gebruik maken van andere technieken.
\\\indent
De vergelijkende studie zal uitgevoerd worden aan de hand van aan tabel met ongestructureerde masterdata afkomstig uit een bestaand bedrijf. Met behulp van bibliotheken in Python zal er onder andere vergeleken worden op basis van de volgende zaken: de hoeveelheid gevonden clusters, grootte van de clusters en de Silhoutte gemiddelde's (hoe goed de waarde thuishoort in zijn cluster) \autocite{Kaplan2022}.

%De inleiding moet de lezer net genoeg informatie verschaffen om het onderwerp te begrijpen en in te zien waarom de onderzoeksvraag de moeite waard is om te onderzoeken. In de inleiding ga je literatuurverwijzingen beperken, zodat de tekst vlot leesbaar blijft. Je kan de inleiding verder onderverdelen in secties als dit de tekst verduidelijkt. Zaken die aan bod kunnen komen in de inleiding~\autocite{Pollefliet2011}:

%\begin{itemize}
%  \item context, achtergrond
%  \item afbakenen van het onderwerp
%  \item verantwoording van het onderwerp, methodologie
%  \item probleemstelling
%  \item onderzoeksdoelstelling
%  \item onderzoeksvraag
%  \item \ldots
%\end{itemize}

\section{\IfLanguageName{dutch}{Probleemstelling}{Problem Statement}}%
\label{sec:probleemstelling}
Masterdata is iets wat quasi alle bedrijven hebben en bijhouden. Deze data is vaak lastig om goed geordend bij te houden. Bij de opstart van een bedrijf, als er nog zeer weinig masterdata is om op te slaan, gaat dit nog gemakkelijk, maar na een aantal jaar is de controle makkelijk te verliezen en verslechterd de datakwaliteit. Zo wordt het moeilijk om in deze data aan data-extractie te doen.


%Uit je probleemstelling moet duidelijk zijn dat je onderzoek een meerwaarde heeft voor een concrete doelgroep. De doelgroep moet goed gedefinieerd en afgelijnd zijn. Doelgroepen als ``bedrijven,'' ``KMO's'', systeembeheerders, enz.~zijn nog te vaag. Als je een lijstje kan maken van de personen/organisaties die een meerwaarde zullen vinden in deze bachelorproef (dit is eigenlijk je steekproefkader), dan is dat een indicatie dat de doelgroep goed gedefinieerd is. Dit kan een enkel bedrijf zijn of zelfs één persoon (je co-promotor/opdrachtgever).

\section{\IfLanguageName{dutch}{Onderzoeksvraag}{Research question}}%
\label{sec:onderzoeksvraag}
Dit onderzoek focust zich op het vergelijken van verschillende mogelijkheden om ongestructureerde masterdata te clusteren. Aan de hand van de resultaten van de vergelijkende studie, kunnen bedrijven een antwoord vinden op volgende onderzoeksvraag:
\begin{itemize}
    \item Wat is de beste manier om ongestructureerde masterdata te clusteren?
\end{itemize}
\\\indent
De resultaten van de vergelijkende studie zullen zodanig opgesteld worden dat er voor verschillende soorten masterdata een goede manier naar voren geschoven kan worden.

%Wees zo concreet mogelijk bij het formuleren van je onderzoeksvraag. Een onderzoeksvraag is trouwens iets waar nog niemand op dit moment een antwoord heeft (voor zover je kan nagaan). Het opzoeken van bestaande informatie (bv. ``welke tools bestaan er voor deze toepassing?'') is dus geen onderzoeksvraag. Je kan de onderzoeksvraag verder specifiëren in deelvragen. Bv.~als je onderzoek gaat over performantiemetingen, dan

\section{\IfLanguageName{dutch}{Onderzoeksdoelstelling}{Research objective}}%
\label{sec:onderzoeksdoelstelling}
Het doel van dit onderzoek is een goede leidraad te voorzien voor bedrijven die orde willen brengen in hun ongestructureerde masterdata en de datakwaliteit verbeteren. Aan de hand van de bekomen resultaten kunnen bedrijven de juiste methode vinden die het beste werkt voor hun data.

%Wat is het beoogde resultaat van je bachelorproef? Wat zijn de criteria voor succes? Beschrijf die zo concreet mogelijk. Gaat het bv.\ om een proof-of-concept, een prototype, een verslag met aanbevelingen, een vergelijkende studie, enz.
\section{\IfLanguageName{dutch}{Opzet van deze bachelorproef}{Structure of this bachelor thesis}}%
\label{sec:opzet-bachelorproef}
% Het is gebruikelijk aan het einde van de inleiding een overzicht te
% geven van de opbouw van de rest van de tekst. Deze sectie bevat al een aanzet
% die je kan aanvullen/aanpassen in functie van je eigen tekst.

De rest van deze bachelorproef is als volgt opgebouwd:

In Hoofdstuk~\ref{ch:stand-van-zaken} wordt een overzicht gegeven van de stand van zaken binnen het onderzoeksdomein, op basis van een literatuurstudie. Ook wordt er een lijst opgesteld van alle mogelijkheden die onderzocht worden.

In Hoofdstuk~\ref{ch:methodologie} wordt de methodologie toegelicht en worden de gebruikte onderzoekstechnieken besproken om een antwoord te kunnen formuleren op de onderzoeksvraag.

% TODO: Vul hier aan voor je eigen hoofstukken, één of twee zinnen per hoofdstuk

In Hoofdstuk~\ref{ch:conclusie}, tenslotte, wordt de conclusie gegeven en een antwoord geformuleerd op de onderzoeksvraag. Daarbij wordt ook een aanzet gegeven voor toekomstig onderzoek binnen dit domein.

















